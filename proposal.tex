\documentclass{article}

\usepackage{xcolor}
\usepackage{comment}
\usepackage{graphicx}
\usepackage{csquotes}
\usepackage{balance}
\usepackage{setspace}

\usepackage{listings}
\usepackage{subcaption}

\usepackage{standalone}


\newcommand{\todo}[1]{{\textcolor{red}{{\tt{TODO:}}\,\,#1 }}}
\newcommand{\nc}[0]{\todo{cite}}
\newcommand{\an}[1]{{\textcolor{blue}{Author's Note: #1}}}
\newcommand{\ttt}[1]{{\texttt{#1}}}

\author{Brandon Neth}
\title{Dissertation Proposal}

\begin{document}

\maketitle

\section{Introduction}
From optimizing wind-farm performance~\cite{sprague2020exawind} to understanding the dynamics of the watershed~\cite{olschanowsky2019hydroframe}\todo{cite parflow},
high-performance computing (HPC) is a critical strategy for combatting the climate crisis. 
When developing such applications, domain scientists often encounter the need to optimize or otherwise refactor their application code. 
Generally, they have two options: learn how to optimize their code themselves or enlist the help of software engineers. 
\todo{write off the enlisting a programmer line in a short sentence.} 
For those that choose to tackle the problem themselves, challenge awaits. 

Peculiars of the application and its implementation consume development time, often for improvements localized to a single machine. 
Moreover, as computers develop and complexify post-Moore's Law, optimal performance hides behind a machine-specific balance of parallelism and data locality that most languages do not explicitly support; and 
different parts of computations need different approaches to optimization, combining schedule transformations with data transformations on both dense and sparse code.
In short, scientists need a concise, portable interface for optimizing their existing codes with varying levels of granularity.

\textbf{I propose to develop a C++ extension that supports user-guided, model-supplemented optimization of loop schedule and program data for existing dense and sparse codes.}
\paragraph{$\dots$ \textit{a C++ extension that} $\dots$}
C++ is a industry standard programming language for high performance computing as well as a common standard for many introductory engineering and science computing courses.

\paragraph{$\dots$ \textit{supports user-guided, model-supplemented optimization} $\dots$}
While some users have an exact idea of the optimizations they want to use for each part of a computation, it should not be required that every optimizing transformation be user-specified. 
Thus, I will expose all transformations as part of a succinct API while also augmenting the optimization process with estimates from a performance model. 
This will enable users to specify as much or as little of the transformations as they like and have the model \enquote{fill in} the rest.

\paragraph{$\dots$ \textit{of loop schedule and program data for} $\dots$}
While loop schedule transformations are often critical to improving data locality and parallelism, data format is also a crucial consideration, especially in sparse codes. 
Any framework that supports only one or the other leaves potentially orders of magnitude performance improvements on the table.
For this reason, I will support both types of transformations in an integrated framework.
\paragraph{$\dots$ \textit{existing sparse and dense codes.}}
Support for sparse and dense codes together means supporting a wider range of applications, especially those that use sparse data in some parts and dense in others.
\section{Overview}

\section{Intellectual Merit}


\section{Broader Impacts}

We are on the brink of climate catastrophe. 
There can be no broader impact than helping avert it.

\section{Research Plan}


\section{Project Milestones}

\begin{table*}[h]
    \centering
\begin{tabular}{|p{2.5cm}|p{5cm}|c|c|p{3cm}|}
    \hline
    \textbf{Title} & Short Description & Start Date & End Date & Deliverable(s)\\
    \hline
    Hydrodynamics \linebreak Chapel Port & Porting key kernels of a hydrodynamics code from C to Chapel & May 2022 & August 2022 & Ported code, report on process \\
    \hline
    \todo{} & & & & \\
    \hline
    Final & Dissertation completed & \todo{} & May 2023 & Defended dissertation\\
    \hline
\end{tabular}

\caption{Milestone overview}
\label{MilestoneTable}
\end{table*}
\section{Preliminary Results}


\section{Data Plan}

All code related to my dissertation will be open source under the Atmosphere Software License, which imposes enforcable divestment obligations on the relicensing of software~\cite{atmospherelicense}. 
\bibliographystyle{abbrv}
\bibliography{proposal}
\end{document}